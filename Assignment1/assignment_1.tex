\documentclass[10pt]{article}
\usepackage{fullpage}
\usepackage{amsmath}

\begin{document}

\noindent published several papers in the 1830s concerning properties of the function $0^{0^x}$. He noted [32] that $0^x$ is either 0 (if $x > 0$) or 1 (if x = 0) or ${\infty}$ (if $x < 0$) hence, 

\begin{align}
    0^{0^x} = [x>0].  
\end{align}

\noindent But of course he didn't have Iverson's convention to work with; he was pleased to discover a way to denote the discontinuous function[$x > 0$] without leaving the realm of operations acceptable in his day. He believed that ``la fonction $0^{0^{x-n}}$ est d'un grand usage dans l'analyse math\`ematique." And he noted in [33] that his formulas "ne renferment aucune notation nouvelle.\ldots Les formules qu'on obtient de cette mani\`ere sont tr\`es simples, et rentrent dans l'alg\`ebre ordinaire."

\\
\indent Libri wrote, for example,
\begin{align*}
    (1 - 0^{0^{-x}})(1 - 0^{0^{x-a}})
\end{align*}

\noindent for the function $[0 \leq x \leq a]$, and he gave the integral formula
\begin{align*}
    \frac{2}{\pi} \int_{0}^{\infty} {\frac{dq\cos qx}{1+q^2}} = e^x \cdot 0^{0^{-x}} + e^{-x} (1+0^{0^{-x}}) = \frac{e^x}{0^{-x}+1} + \frac{e^{-x}}{0^x + 1}.
\end{align*}

\noindent (Of course, we would now write the value of that integral as $e^{-|x|}$, but a simple notation for absolute value wasn’t introduced until many years later. I believe that the first appearance of `$|z|$'for absolute value in Crelle’s journal - the journal containing Libri’s papers [32] and [33] - occurred
on page 227 of [56] in 1881. Karl Weierstrass was the inventor of this notation, which was applied at first only to complex numbers; Weierstrass seems to have published it first in 1876 [55].)
\\
\indent Libri applied his $0^{0^{x}}$ function to number theory by exhibiting a complicated way to describe the fact that $x$ is a divisor of $m$.  In essence, he gave the following recursive formulation: Let $P_0(x) = 1$ and for $k > 0$ let

\begin{align*}
    P_k(x) = 0^{0^{x-k}} P_0(x) - 0^{0^{x-k+1}} P_1(x) - \ldots - 0^{0^{x-1}} P_{k-1}(x).
\end{align*}
Then the quantity

\begin{align*}
    \frac{1 - m \cdot 0^{0^{x-k}}P_0(x) - (m-1) 0^{0^{x-k+1}} P_1(x) - \ldots -2 \cdot 0^{0^{x-2}} P_{m-2}(x) - 0^{0^{x-1}} P_{m-1}(x)}{x}
\end{align*}
\\
turns out to equal 1 if $x$ divides $m$ otherwise it is 0. (One way to prove this, Iverson-wise, is to replace $0^{0^{x-k}}$ in Libri’s formulas by $[x>k]$, and to show first by induction that $P_k(x)$ = [$x$ divides $k$] - [$x$ divides $k-1$] for all $k > 0$. Then if $a_k(x)$ = $k[x>k]$, we have

\begin{align*}
        \sum_{k=0}^{m-1} a_{m-k}(x)P_k(x) = \sum_{k=0}^{m-1}a_{m-k}(x)([x {\:} {\text{divides} {\:}} k] - [x {\:} {\text{divides}} {\:} k - 1]) = \sum_{k=0}^{m-1} [x {\:} {\text{divides}} {\:} k] (a_{m-k}(x) - a_{m-k-1}(x))
\end{align*}
\\
If the positive integer $x$ is not a divisor of $m$,  the terms of this new sum are zero except when $m-k$ = $m$ mod $x$ when we have $a_{m-k}(x) - a_{m-k-1}(x)) = 1$. On the other hand if $x$ is a divisor of $m$, the only nonvanishing term occurs for $m-k = x$, when we have $a_{m-k}(x) - a_{m-k-1}(x) = 0 - (x-1)$. Hence the sum is $1-x[x$ divides $m]$. Libri obtained his complicated formula by a less direct method, applying Newton's identities to compute the sum of the $m$th powers of the roots of the equation $t^{x-1} + t^{x-2} + \ldots + 1 = 0$.)
\\
\indent Evidently Libri's main purpose was to show that unlikely functions can be expressed in algebraic terms, somewhat as we might wish to show that some complex functions can be computed by a Turing Machine. ``Give me the function $0^{0^{x}}$, and I'll give you an expression for [$x$ divides $m$]." But our goal with Iverson's notation is, by contrast, to find a simple and natural way to express quantities that help us solve problems. If we need a function that is 1 if and only if $x$ divides $m$, we can now write [$x$ divides $m$].
\\
\indent Some of Libri's papers are still well remembered, but [32] and [33] are not. I found no mention of them in {\em{Science Citation Index}},  after searching through all years of that index available in our library (1955 to date). However, the paper [33] did produce several ripples in mathematical waters when it originally appeared, because it stirred up a controversy about whether $0^0$ is defined. Most mathematicians agreed that $0^0$ = 1, but Cauchy [5, page 70] had listed $0^0$ together with other expressions like 0/0 and $\infty$ - $\infty$ in a table of undefined forms. Libri's justification for the equation $0^0$ = 1  was far from convincing, and a commentator who signed his name simply "S" rose to the attack [45].  August M\"obius [36] defended Libri, by presenting his former professor's reason for believing that $0^0 = 1$  (basically a proof that $\lim_{x\to 0} + x^x = 1$). M\"obius also went further and presented a supposed proof that $\lim_{x\to 0} + f(x)^{g(x)} = 1 $ whenever $\lim_{x\to 0} + f(x) = \lim_{x\to 0} + g(x) = 0$. Of course the ``S"  then asked [3] whether M\"obius knew about functions such as $f(x) = e^{-1/x}$ and $g(x) = x$.  (And paper [36] was quietly omitted from the historical record when the collected works of M\"obius were ultimately published.) The debate stopped there, apparently with the conclusion that $0^0$ should be undefined.
\\
\indent But no, no, ten thousand times no! Anybody who wants the binomial theorem

\begin{align}
    (x+y)^{n} = \sum_{k=0}^{n} \binom{n}{k} x^k y^{n-k}
\end{align}

\noindent to hold for at least one nonnegative integer $n$ {\em{must}} believe that $0^0 = 1$, for we can plug in $x = 0$ and $y=1$ to get 1 on the left and $0^0$ on the right.
\\
\indent The number of mappings from the empty set to the empty set is $0^0$. It {\em{has}} to be 1.
\\
\indent On the other hand, Cauchy had good reason to consider $0^0$ as an undefined {\em{limiting form}},  in the sense that the limiting value of $f(x)^{g(x)}$ is not known a {\em{priori}} when $f(x)$ and $g(x)$ approach 0 independently. In this much stronger sense, the value of $0^0$ is less defined than, say, the value of $0+0$. Both Cauchy and Libri were right, but Libri and his defenders did not understand why truth was on their side.
\\
\indent Well, it's instructive to study mathematical history and to observe how tastes change as progress is made. But let's come closer to the present, to see how Iverson's convention might be useful nowadays. Today's mathematical literature is, in fact, filled with instances where analogs of Iversonian brackets are being used - but the concepts must be expressed in a roundabout way, because his convention is not yet established. Here are two examples that I happened to notice
\end{document}

