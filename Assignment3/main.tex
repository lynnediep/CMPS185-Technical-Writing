\documentclass[twocolumn]{article}
\usepackage[utf8]{inputenc}
\usepackage{fullpage}

\title{Assignment 3}
\author{Lynne Diep }
\date{October 20, 2017}

\begin{document}

\maketitle

\section{ACM - ACM Transactions on Algorithms (TALG)}
The publisher of this journal is ACM and the name of the journal is ACM Transactions on Algorithms (TALG). Here is the URL for the journal: 
\\\texttt{https://dl.acm.org/pub.cfm?id=J982&CFID=9
95746199&CFTOKEN=54523148}
\\
TALG is a scholarly journal that focuses on discrete and finite algorithms. Most journals that are published feature new algorithms, new analyses, improved data structures, and more along those lines. TALG features the works of 54 ACM fellows, including Robert Andre Tarjan, the winner of the 1986 A. M. Turing Award. Specific areas covered by the journal include: parallel and distributed computation, data compression, cryptography, and other related areas in the scope as well.
The Editor-in-Chief is Aravind Srinivasan and the size of the Editorial Board is 27.
The journal is published quarterly as well.
Submissions for TALG is done electronically through 
\texttt{http://mc.manuscriptcentral.com/talg}. Manuscripts must be standard journal formatting, including an abstract, introduction, technical sections, and a bibliography. The file must be in either {\LaTeX} or Word, and the proper indexing and retrieval information must be provided from the ACM Computing Classification System.
The journal does not have double-blind reviewing policy and page limits for submitted articles.
The journal has the option for open access publishing, which leaves all rights to the author and grants a non-exclusive license to ACM for publication.
What I find interesting about TALG is its legitimacy. There are many authors who have won awards for their publications, and the download count is fairly high. The publication process is also strict and exclusive, so the reader knows each publication in the journal is accurate and professional.

\section{IEEE - IEEE Transactions on Multi-Scale Computing Systems (TMSCS)} 
The publisher of this journal is IEEE Computer Society and the name of the journal is IEEE Transactions on Multi-Scale Computing Systems (TMSCS). Here is the URL for this journal:
\\\texttt{https://www.computer.org/web/tmscs}
\\
TMSCS is a publication the focuses on computing systems that exploit multi-scale and multi-functionality. Topics include: efficient algorithms for information distribution/processing, hardware and software solutions for IoT applications, new and emerging application areas and computing trends, and other areas related within this scope.
The Editor-in-Chief of the journal is Partha Pratim Pande  and the size of the Editorial Board is 20.
This journal is published quarterly as well.
To submit a paper to TMSCS, it must be submitted to ScholarOne Manuscripts. There must be an abstract that states the significance of the paper, keywords, figures and tables, and other factors in any typical scholarly journal. For ScholarOne Manuscripts, follow the instructions on how to submit the manuscript and fill out the needed information.
For TMSCS, here are the page limits:
\\Regular paper – 14 double column pages 
\\Short paper – 8 double column pages
\\Comments paper – 2 double column pages
\\Survey paper – 20 double column pages
\\
The journal does have double-blind review policy and open access publishing, which leaves all rights to the author and grants a non-exclusive license to IEEE for publication.
What I find interesting about TMSCS is the magnitude it has. IEEE is a extremely reliable publisher, and any journal that it publishes automatically has the credibility compared to other scholarly journals.

\section{ACM Journal Titles}
\subsection{Nearly Optimal Deterministic Algorithm for Sparse Walsh-Hadamard Transform}
The title clearly states what the journal will be about; however, the word ``nearly" does not suit well with me. It makes me think that the journal is not complete and will not have a strong conclusion. Regardless, I would still use this publication as a resource if I was studying this area.
\subsection{Hollow Heaps}
This title is short and automatically makes me wonder, ``What is/are Hollow Heaps?" I would definitely click on the publication to find out more, and learn about this new data structure.
\subsection{Tight Kernel Bounds for Problems on Graphs with Small Degeneracy }
This title is exact and gets straight to the point. I have no questions regarding the title, and I assume the publication is what the title states. The title is specific, and I would say that the title is fine how it is now.

\section{IEEE Journal Titles}
\subsection{Exploiting the Potential of Computation Reuse Through Approximate Computing}
This title is also gives an exact point, and I would assume the journal will focus on what the title states. If I were doing research within this area, I would probably use this publication as a resource based on the title alone.
\subsection{Introduction to Cyber-Physical System Security: A Cross-Layer Perspective}
I feel like the title is not specific enough for a scholarly journal, and there are plenty of publications about Cyber Security. Would most likely use this journal for my introduction on cyber security, and not use it for a conference paper.
\subsection{On The Outside Looking In: Towards Detecting Counterfeit Devices Using Network Traffic Analysis}
This title is more specific, and I get a sense of what the main idea of the journal is. I would personally remove ``On The Outside Looking In:" part and leave that title as ``Detecting Counterfeit Devices Using Network Traffic Analysis".

\begin{thebibliography}{9}
\bibitem{Cheraghchi}
Mahdi Cheraghchi and Piotr Indyk. 
\textit{Nearly Optimal Deterministic Algorithm for Sparse Walsh-Hadamard Transform}. 
ACM Trans. Algorithms 13, 3, Article 34 (March 2017), 36 pages. DOI: https://doi.org/10.1145/3029050

\bibitem{Hollow}
Thomas Dueholm Hansen, Haim Kaplan, Robert E. Tarjan, and Uri Zwick.
\textit{Hollow Heaps}. 
ACM Trans. Algorithms 13, 3, Article 42 (July 2017), 27 pages. DOI: https://doi.org/10.1145/3093240

\bibitem{TightKernel}
Marek Cygan, Fabrizio Grandoni, and Danny Hermelin.
\textit{Tight Kernel Bounds for Problems on Graphs with Small Degeneracy}. 
ACM Trans. Algorithms 13, 3, Article 43 (August 2017), 22 pages. DOI: https://doi.org/10.1145/3108239

\bibitem{XinHe}
Xin He, Shuhao Jiang, Wenyan Lu, Guihai Yan, Yinhe Han, and Xiaowei Li. 
\textit{Exploiting the Potential of Computation Reuse Through Approximate Computing}.
IEEE Transactions on Multi-Scale Computing Systems vol. 3 no. 3. p. 152-165. 2017.

\bibitem{JacobWurm}
Jacob Wurm, Yier Jin, Yang Liu, Shiyan Hu, Kenneth Heffner, Fahim Rahman, and Mark Tehranipoor.
\textit{Introduction to Cyber-Physical System Security: A Cross-Layer Perspective}. 
IEEE Transactions on Multi-Scale Computing Systems vol. 3 no. 3. p. 215-227. 2017.

\bibitem{Reece}
T. Reece, S. Sathyanarayana, W. H. Robinson and R. A. Beyah.
\textit{On The Outside Looking In: Towards Detecting Counterfeit Devices Using Network Traffic Analysis}.
IEEE Transactions on Multi-Scale Computing Systems. vol. 3 no. 1. pp. 50-61. Jan.-March 1 2017.
\end{thebibliography}
\end{document}
